\sectionnonum{Įvadas}

\subsectionnonum{Temos aktualumas}
Tobulėjant programinės įrangos (toliau PĮ) kūrimo įrankiams ir vis augant informacinių sistemų sudėtingumo poreikiams, 
kyla daug klausimų PĮ kūrėjams, kokį architektūros modelį ir kokias technologijas pasirinkti, pradedant kurti naują informacinę sistemą.
Vis daugėja skirtingų technologijų ir programavimo kalbų, kurios yra pranašesnės už kitas tik siaurose srityse,
dėl to dažnu atveju neužtenka pasirinkti vieną technologiją ar programavimo kalbą norint sukurti kokybišką ir
tvarią PĮ. Kai kurios technologijos yra pranašesnės resursų taupyme, kitos pranašesnės daug skirtingų bibliotekų palaikymu ir lengvai naudojama
programų (angl. \textit{„application“}) programavimo sąsaja (angl. \textit{„Application programming interface“} arba \textit{„API“}) ir t.t.
Kuriant naują PĮ reikia gerai išsianalizuoti tuo metu esamas technologijas ir jų privalumus.
Įmonės yra linkusios kurti PĮ tokiomis technologijomis, kurių specialistų yra daug ir kurie norėtų palaikyti ir kurti jomis.
Renkantis skirtingas technologijas, atsiranda problema, kaip jas apjungti, kad veiktų vieningai.
Tokiu atveju, galima naudotis saitynų tarnybų pagalba (angl. "Web services"), tačiau to neužtenka,
nes norima skirtingus funkcionalumus įgyvendinti skirtingų technologijų pagalba.
Tada dažnu atveju kuriamos mikroservisų (angl. \textit{„Microservices“}) sistemos. Remiantis šiuo modeliu būtų kuriami atskiri moduliai (taip pavadinsime atskirus sitemos vienetus), kurių
kiekvienas būtų atsakingas už tam tikrą funkcionalumą. Problema ta, kad moduliai yra skirtingi, tačiau jie turi sąveikauti tarpusavyje, o tai pasiekti gali būti sudėtinga.  
\subsectionnonum{Problema}
Kokį integracijų tipą rinktis mikroservisų architektūrose, norint įgyvendinti komunikaciją tarp skirtingų tarnybų (angl. \textit{„service“})?
\subsectionnonum{Darbo tikslas}
Palyginti skirtingus komunikacijų tipus, apžvelgti jų pranašumus ir trūkumus. Pateikti situacijų, kur vieni tipai yra pranašesni už kitus, pavyzdžių 
ir technologijų, šiems tipams realizuoti.
\subsectionnonum{Uždaviniai tikslui pasiekti}
\begin{enumerate}
	\item Remiantis literatūra apibūdinti, kas yra mikroservisai, kuo jie ypatingi, kaip jie projektuojami.
	\item Išskirti pagrindinius integracijų ir komunikavimo tipus ir jų savybes.
	\item Palyginti skirtingus komunikavimo būdus mikroservisų architektūrose.
	\item Pateikti konkrečius komunikavimo integracijų ir technologijų pavyzdžius.
	\item Pateikti rekomendacijas, kokiais atvejais, kokius tipus būtų geriau naudoti.
\end{enumerate}

