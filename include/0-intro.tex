\sectionnonum{Įvadas}

\subsectionnonum{Temos aktualumas}
Tobulėjant programinės įrangos (toliau PĮ) kūrimo įrankiams ir vis augant informacinių sistemų sudėtingumo poreikiams,
kyla daug klausimų PĮ kūrėjams kokį architektūros modelį ir kokias technologijas pasirinkti, pradedant kurti naują informacinę sistemą.
Vis daugėja skirtingų technologijų ir programavimo kalbų, kurios yra pranašesnes už kitas tik siaurose srityse,
dėl to dažnu atveju neužtenka pasirinkti vieną technologiją ar programavimo kalbą norint sukurti kokybišką ir
tvarią PĮ. Kai kurios technologijos yra pranašesnės resursų taupyme, kitos pranašesnės daug skirtingų bibliotekų palaikymu ir lengvai naudojama
aplikacijų programavimo sąsają (angl. \textit{„Application programming interface“} arba \textit{„API“}) ir t.t.
Kuriant naują PĮ reikia gerai išsianalizuoti tuo metu esamas technologijas ir jų privalumus.
Įmonės yra linkusios kurti PĮ tokiomis technologijomis, kurių specialistų yra daug ir kurie norėtų palaikyti ir kurti jomis.
Renkantis skirtingas technogolijas, atsiranda problema kaip jas apjungti, kad veiktų vieningai.
Tokiu atveju, galima naudotis saitynų tarnybų pagalba (angl. "Web services"), tačiau to neužtenka,
nes kas, jeigu norime, skirtingus funkcionalumus įgyvendinti skirtingų technologijų pagalba.
Tada dažnu atveju naudojamas mikroservisų (angl. \textit{„Microservices“}) architektūrinis modelis. Remiantis šiuo modeliu būtų kuriami atskiri moduliai (taip pavadinsime atskirus sitemos vienetus), kurių
kiekvienas būtų atsakingas už savo funkcionalumą. Problema su šia architektūra, kad reikia priversti šiuos skritingus modulius bendrauti tarpusavyje, o tai nebūna taip paprasta.
\subsectionnonum{Problema}
Kokį integracijų tipa rinktis mikroservisų architektūrose, norint įgyvendinti komunikaciją tarp skirtingų servisų.
\subsectionnonum{Darbo tikslas}
Palyginti skirtingus komunikacijų tipus, apžvelgti jū pranašumus ir trūkumus, pateikti pavyzdžių, kur vieni tipai yra pranašesni už kitus, 
ir pateikti technologijų šiem tipam realizuoti pvyzdžių.
\subsectionnonum{Uždaviniai tikslui pasiekti}
\begin{enumerate}
	\item Remiantis literatūra apibūdinti, kas yra mikroservisai, kuo jie ypatingi, kaip jie projektuojami.
	\item Išskirti pagrindinius integracijų ir komunikavimo tipus ir jų savybes.
	\item Palyginti skirtingus komunikavimo būdus mikroservisų architektūrose.
	\item Pateikti konkrečius komunikavimo integracijų ir technologijų pavyzdžius.
	\item Pateikti rekomendacijas, kokiais atvejais, kokius tipus būtų geriau naudoti.
\end{enumerate}

