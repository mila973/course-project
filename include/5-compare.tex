\section{Skirtingų integracijų tipų privalumai ir trūkumai}


\subsection{Silpnas sujungimas ir stiprus sanglauda (angl. \textit{„Loose Coupling and High Cohesion“})}
Sinchronines ir asinchronines integracijas mikroservisuose galima palyginti pagal tai ar tai atitinka pagrindinius mikroservisų
kriterijus. Pagrindiniai principai, pagal kuriuos galima teigti, kad tarnyba yra tinkamai suprojektuota yra \cite{Bk2}:

\begin{itemize}
  \item Silpnas sujungimas (angl. \textit{„Loose Coupling“}).Tai reiškia, kad kiekviena tarnyba turėtų kuo labiau būti nepriklausoma. Jeigu yra daromas pakeitimas tarnybai, tai turėtų
  tik jai vienai ir būti atliekama, nes vienas is pagrindinių mikroservisų privalumų yra galimybė įdiegti vieną tarnybą, nediegiant kitų.
  \item Stiprus sąryšis (anlg. \textit{„High Cohesion“}).Tai reiškia, kad kievkiena tarnyba tūrėtų savo logiką laikyti 
  pas save ir atlikti visus su savo atsakomybėmis susijusius procesus.
\end{itemize}

Šiuose dviejuose aspektuose ryškus asinchroninių integracijų pranašumas. Asinchroninių komunikacijų metu, tarnybos tiesiog siunčia pranešimus apie procesų pabaigą ir 
klausosi, kada juos pradėti. Sinchroninės komunikacijos reikalauja žinoti implementacijos detales ir kreiptis į kitas tarnybas, per klientus arba tiesiogiai.
Dažniausiai asinchroninės komunikacijos visoje sistemoje veikia vienodai.

\subsection{Efektyvumas}

Vienas pagrindinių skirtumų tarp sinchroninių ir asinchroninių integracijų yra blokavimas. Kaip jau minėta ankščiau,
sinchroninės integracijos yra blokuojančios, priešingai nei asinchroninės. Kaip minima įmonės „Microsoft“
publikuotame darbe apie mikroservisus ir jų kūrimą su .NET technologija \cite{Misc1} vykdant ilgus ir daug resursų reikalaujančius procesus
pasitelkiant tokią technologiją kaip RESTful, dėl ilgo atsakymo laukimo galima patirti operacijos laiko pasibaigimus (angl. \textit{„timeouts“})
ir dėl to sėkmingai neužbaigti procesų.

Kitas asinchroninių integracijų privalumas yra lygiagretiškumas. Priešingai nei sinchroniniai komunikavimai,
pranešimus gali prenumeruoti ir vienu metu savo pareigybes vykdyti keli procesai, kas, žinoma, padidina sistemos
efektyvumą ir sutrumpina procesų veikimo laiką.

\subsection{Implementacijos kompleksiškumas}

Dar vienas labai svarbus faktorius renkantis tarp technologijų yra kompleksiškumas. Kuriant programinę įrangą kompleksiški reikalavimai
reiškia, kad atsiras daug vietos klaidoms. Klaidos labai stipriai didina gamybos kaštus ir sistemos vartotojų nepasitenkinimą.
Asinchroninės integracijos yra žymiai sudėtingesnės ir dažniausiai naudojami pranešimų brokeriai reikalauja papildomų 
implementacijos detalių, kad pavyktų sujungti tarnybą su brokeriu ir pavyktų klausytis arba publikuoti pranešimus.
Tokie darbai iš PĮ kūrėjų reikalauja aukštos kvalifikacijos, o tokių darbuotojų, kurie atitiktų reikalavimus,  įmonėms yra daug sunkiau surasti ir įdarbinti.
Kita vertus sinchroniniai tarnybų kvietimai yra paprasti ir tarnybų programų programavimo sąsajos būna išlaikomos primityvios ir trivialios.
Tas minima ir David S.Linthicum knygoje „Enterpise Application Integration“ \cite{Bk3}.

\subsection{Veiksmų istorija}

Kuriant dideles informacines sistemas, dažnai reikia atsižvelgti į įvykių registravimą. Dažnas reikalavimas didelėms sistemoms yra veiksmų istorija.
Tokiems uždaviniams spręsti geriausiai tinka asinchroninės užklausos, nes pasitelkus tokias technologijas kaip pranešimų brokeriai,
žinutės ir įvykiai būna saugomi, ypač įvykiais paremtose (angl. \textit{„event-based“}) architektūrose.
Sinchroninių integracijų metu įvykių istorijos problemą reikia spręsti kitaip ir ji netampa tokia triviali.

\subsection{Įvykių sekos užtikrinimas}

Procesų vykdymai, kaip jau minėta studentų registracija, neretai reikalauja užtikrinimo, kad vienas procesas bus įvykdomas ankščiau už kitą.
Sinchroninės integracijos yra blokuojamos ir vykdomos paeiliui, todėl šios problemos nelieka, tačiau asinchroninis komunikavimas gali būti 
lygiagretinamas ir siekiant užtikrinti korektišką eiliškumą reikia įgyvendinti papildomą logiką. Dėl tokių atvejų, lygiagretūs procesai
tampa dar labiau komplikuoti. Tokiais atvejais asinchroninio komunikavimo metu tarnybos turi saugoti proceso būseną ir užtikrinti veiksmų eiliškumą.

\subsection{Tarnybų kompozicija}

Vienas iš mikroservisų architektūros privalumų yra išskaidytų tarnybų perpanaudojimas. Įmonės, kurios užsiima projektiniais darbais,
neretai siekia kurti tokią PĮ, kuria būtų galima pasiremti arba jos komponentes panaudoti ateities projektuose. Mikroservisai, kurie naudoja 
sinchronines integracijas būna mažiau surišti su domeno logika ir norint integruoti tarnybą kitoje informacinėje sistemoje
neprivaloma taip prisirišti prie technologijų. Tokios tarnybos įmonėms atneša daug naudos ir sutaupo kaštų ateityje.


\subsectionnonum{Tyrimas}

Toliau pateikta lentelė su lyginimo rezultatais tarp sinchroninių komunikacijų ir asinchroninių remiantis šiame skyrelyje apibrėžtais kriterijais ir analize (\ref{tab:research} lentelė). Prie komunikavimo tipo
surašyti kriterijai, kuriuose nurodytas komunikavimo tipas yra pranašesnis:

\begin{table}[H]
  \caption{Komunikavimo tipų palyginimo rezultatai.}
  \label{tab:research}
  \begin{tabular}{ |p{7.8cm}|p{7.8cm}|  }
    \hline
      Asinchroninis komunikavimas & Sinchroninis komunikavimas\\
    \hline
    Silpnas sujungimas ir stiprus sanglauda & Implementacijos kompleksiškumas \\
    Efektyvumas & Įvykių sekos užtikrinimas \\
    Veiksmų istorija  & Tarnybų kompozicija \\
    \hline
  \end{tabular}
\end{table}