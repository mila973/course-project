\usepackage{algorithmicx}
\usepackage{algorithm}
\usepackage{algpseudocode}
\usepackage{amsfonts}
\usepackage{amsmath}
\usepackage{bm}
\usepackage{caption}
\usepackage{color}
\usepackage{float}
\usepackage{graphicx}
\usepackage{listings}
\usepackage{subfig}
\usepackage{wrapfig}
\usepackage[backend=biber]{biblatex}
\usepackage[table,xcdraw]{xcolor}
\usepackage{booktabs}

\usepackage{enumitem}

\setitemize{noitemsep,topsep=0pt,parsep=0pt,partopsep=0pt}
\setenumerate{noitemsep,topsep=0pt,parsep=0pt,partopsep=0pt}

\setmainfont{Palemonas}

\university{Vilniaus universitetas}
\faculty{Matematikos ir informatikos fakultetas}
\department{Informatikos studijų programa}
\papertype{Kursinis darbas}
\title{Pagrindiniai komunikavimo integracijų mikroservisų architektūrose tipai ir jų analizė}
\titleineng{Main communication integration types in microservices architectures comparison and analysis}
\status{3 kurso 4 grupės studentas}
\author{Lukas Milašauskas}
\supervisor{Dr. Saulius Minkevičius}
\date{Vilnius – \the\year}

\bibliography{bibliografija}
