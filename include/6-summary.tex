\section{Rezultatai ir išvados}

Šiame darbe buvo išanalizuoti pagrindiniai komunikavimo integracijų mikroservisų architektūrose tipai, pateiktos mikroservisų
atsiradimo priežastys ir aptarti mikroservisų architektūrų pranašumai prie monolitines sistemas. Taip pat įvardinta tinkamo integracijų 
pasirinkimo svarba. Buvo išskirti pagrindiniai mikroservisų komunikavimo tipai:

\begin{enumerate}
	\item Servisų jungimas per duomenų bazę.
	\item Sinchroninės užklausos/atsakymo (angl. \textit{„request/response“}) integracijos.
	\item Asinchroninės įvykiais paremtos (angl. \textit{„event-based“}) integracijos.
\end{enumerate}

Darbe buvo detaliau aptarti sinchroniniai ir asinchroniniai komunikavimai, tačiau taip pat
paminėtas ir apibūdintas komunikavimas per duomenų bazę tipas.
Apžvelgti dviejų pagrindinių tipų principai. Išskirti skirtingų tipų komunikavimo privalumai ir trūkumai.
Apibūdintos integracijų tipų panaudojimas modelinėse situacijose. Pabaigoje pateiktos pragrindiniai pasirinkimo tarp skirtingų technologijų
aspektai ir priežastys.
\break
Pasinaudojus pateikta informacija, buvo įgyvendinti pagrindiniai darbo uždaviniai:

\begin{itemize}
	\item Apibūdinti, kas yra mikroservisai, kuo jie ypatingi, kaip jie projektuojami, remiantis šaltiniais.
	\item Išskirti pagrindinius integracijų ir komunikavimo tipus ir jų savybes.
	\item Palyginti skirtingus komunikavimo būdus mikroservisų architektūrose.
	\item Pateikti rekomendacijas, kokias atvejais kokį tipą geriausia naudoti.
\end{itemize}

Darbo uždaviniams įgyvendinti buvo naudojamos ne tik teorinės žinios iš įvairių šaltinių, tačiau ir istoriniai faktai, 
padėję paaiškinti tam tikrus mikroservisų ir jų komunikavimo aspektus.
\break
Pateikus darbo rezultatus galima suformuluoti tokias išvadas:

\begin{itemize}
  \item Mikroservisų architektūrose pasirinkimas tarp skirtingų komunikavimo tipų yra labai svarbus, nes tai lemia sitemos veikimo stilių ir daug kitų aspektų.
	\item Sinchroninės integracijos yra paprastesnės ir legviau įgyvendinamos. Jos pranašesnės siekiant greitai gauti rezultatus, be kurių, nebūtų galima vykdyti
	sekančių veiksmų informacinėje sistemoje.
	\item Asinchroninės integracijos yra technologiškai pranašesnės ir žymiai geriau atitinką pagrindinius mikroservisų aspektus, tokius kaip: silpnas sujungimas ir stiprus sąryšis.
	\item Asinchroninės integracijos yra efektyvesnės už sinchronines integracijas, dėl savo lygiagretinimo savybių.
	\item Sinchroninis komunikavimas yra sklandesnis ir paprastesnis, dėl to paliekama mažiau vietos klaidoms.
	\item Iš įmonių pusės daug naudingiau yra kurti sinchronines sistemas dėl kaštų.
\end{itemize}

Taigi darytina išvada, jog projektuojant mikroservisų architektūromis paremtą PĮ yra labai svarbu atsižvelgti į sistemos poreikius ir gerai apsvarstyti, kokiu komunikavimo
stiliumi bus pasitikima. Atsižvelgiant į suformuluotas išvadas, galima teigti, kad nors asinchroninės komunikacijos ne visada yra lengviausias būdas, tačiau 
tokia komunikacija apibūdina geriausias mikroservisų praktikas ir paradigmas.

