\section{Rezultatai ir išvados}

Šiame darbe buvo išanalizuoti pagrindiniai komunikavimo integracijų mikroservisų architektūrose tipai, pateiktos mikroservisų
atsiradimo priežastys ir aptarti mikroservisų architektūrų pranašumai prieš monolitines sistemas. Taip pat įvardinta tinkamo integracijų 
pasirinkimo svarba. Buvo išskirti pagrindiniai mikroservisų komunikavimo tipai:

\begin{enumerate}
	\item Tarnybų jungimas per duomenų bazę.
	\item Sinchroninės užklausos/atsakymo (angl. \textit{„request/response“}) integracijos.
	\item Asinchroninės, įvykiais paremtos (angl. \textit{„event-based“}) integracijos.
\end{enumerate}

Darbe buvo detaliau aptarti sinchroniniai ir asinchroniniai komunikavimai, tačiau taip pat
paminėtas ir apibūdintas komunikavimas per duomenų bazę tipas.
Apžvelgti dviejų pagrindinių tipų principai. Išskirti skirtingų tipų komunikavimo privalumai ir trūkumai.
Apibūdintos integracijų tipų panaudojimas modelinėse situacijose. Pabaigoje pateikti pragrindiniai pasirinkimo tarp skirtingų technologijų
aspektai ir priežastys.
\break
Pasinaudojus pateikta informacija, buvo įgyvendinti pagrindiniai darbo uždaviniai:

\begin{itemize}
	\item Apibūdinta, kas yra mikroservisai, kuo jie ypatingi, kaip jie projektuojami, remiantis šaltiniais.
	\item Išskirti pagrindiniai integracijų ir komunikavimo tipai ir jų savybės.
	\item Remiantis šaltiniais išdėstyti trūkumai ir privalumai apibrėžtuose kriterijuose.
	\item Palyginti skirtingi komunikavimo būdai mikroservisų architektūrose.
	\item Pateiktos rekomendacijas ir scenarijai, kokiomis aplinkybėmis kokį tipą geriausia naudoti.
\end{itemize}

Darbo uždaviniams įgyvendinti buvo naudojamos ne tik teorinės žinios iš įvairių šaltinių, tačiau ir istoriniai faktai, 
padėję paaiškinti tam tikrus mikroservisų ir jų komunikavimo aspektus.
\break
Pateikus darbo rezultatus galima suformuluoti tokias išvadas:

\begin{itemize}
  \item Mikroservisų architektūrose pasirinkimas tarp skirtingų komunikavimo tipų yra labai svarbus, nes tai lemia sistemos veikimo stilių ir daug kitų aspektų.
	\item Sinchroninės integracijos yra paprastesnės ir legviau įgyvendinamos. Jos pranašesnės siekiant greitai gauti rezultatus, be kurių, nebūtų galima vykdyti
	sekančių veiksmų informacinėje sistemoje.
	\item Asinchroninės integracijos žymiai geriau atitinka pagrindinius mikroservisų aspektus, tokius kaip: silpnas sujungimas ir stiprus sąryšis.
	\item Asinchroninės integracijos yra efektyvesnės už sinchronines integracijas, dėl savo lygiagretinimo savybių.
	\item Asinchroninės komunikacijos gali įgyvendinti įvykiais paremtas architektūras, kuriose nereikėtų papildomų pastangų, siekiant gauti informacinės sistemos veiksmų istoriją.
	\item Sinchroninis komunikavimas yra sklandesnis ir paprastesnis, dėl to paliekama mažiau vietos klaidoms.
	\item Sinchroninis tarnybų komunikavimas suteikia lankstesnes galimybes tarnybų kompozicijai, perpanaudojimui.
	\item Iš įmonių pusės daug naudingiau yra kurti sinchronines sistemas dėl kaštų skiriamų darbuotojų kompetencijoms.
\end{itemize}

Taigi darytina išvada, jog projektuojant mikroservisų architektūromis paremtą PĮ yra labai svarbu atsižvelgti į sistemos poreikius ir gerai apsvarstyti, kokiu komunikavimo
stiliumi bus pasitikima. Atsižvelgiant į suformuluotas išvadas, galima teigti, kad nors asinchroninės komunikacijos ne visada yra lengviausias būdas, tačiau 
tokią komunikaciją apibūdina geriausias mikroservisų praktikas ir paradigmas.

