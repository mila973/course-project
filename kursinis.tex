\documentclass{VUMIFPSkursinis}

\usepackage{algorithmicx}
\usepackage{algorithm}
\usepackage{algpseudocode}
\usepackage{amsfonts}
\usepackage{amsmath}
\usepackage{bm}
\usepackage{caption}
\usepackage{color}
\usepackage{float}
\usepackage{graphicx}
\usepackage{listings}
\usepackage{subfig}
\usepackage{wrapfig}
\usepackage[backend=biber]{biblatex}
\usepackage[table,xcdraw]{xcolor}
\usepackage{booktabs}

\usepackage{enumitem}

\setitemize{noitemsep,topsep=0pt,parsep=0pt,partopsep=0pt}
\setenumerate{noitemsep,topsep=0pt,parsep=0pt,partopsep=0pt}

\setmainfont{Palemonas}

\university{Vilniaus universitetas}
\faculty{Matematikos ir informatikos fakultetas}
\department{Informatikos studijų programa}
\papertype{Kursinis darbas}
\title{Pagrindiniai integracijų projektavimo mikroservisų architektūrose tipai ir jų analizė}
\titleineng{Main integration types in microservices architectures and their analysis}
\status{3 kurso 4 grupės studentas}
\author{Lukas Milašauskas}
\supervisor{Dr. Saulius Minkevičius}
\date{Vilnius – \the\year}

\bibliography{bibliografija}


\begin{document}


\maketitle
\cleardoublepage\pagenumbering{arabic}
\setcounter{page}{2}


\tableofcontents
\clearpage

\sectionnonum{Įvadas}

\subsectionnonum{Temos aktualumas}
Tobulėjant programinės įrangos (toliau PĮ) kūrimo įrankiams ir vis augant informacinių sistemų sudėtingumo poreikiams, 
kyla daug klausimų PĮ kūrėjams, kokį architektūros modelį ir kokias technologijas pasirinkti, pradedant kurti naują informacinę sistemą.
Vis daugėja skirtingų technologijų ir programavimo kalbų, kurios yra pranašesnės už kitas tik siaurose srityse,
dėl to dažnu atveju neužtenka pasirinkti vieną technologiją ar programavimo kalbą norint sukurti kokybišką ir
tvarią PĮ. Kai kurios technologijos yra pranašesnės resursų taupyme, kitos pranašesnės daug skirtingų bibliotekų palaikymu ir lengvai naudojama
programų (angl. \textit{„application“}) programavimo sąsaja (angl. \textit{„Application programming interface“} arba \textit{„API“}) ir t.t.
Kuriant naują PĮ reikia gerai išsianalizuoti tuo metu esamas technologijas ir jų privalumus.
Įmonės yra linkusios kurti PĮ tokiomis technologijomis, kurių specialistų yra daug ir kurie norėtų palaikyti ir kurti jomis.
Renkantis skirtingas technologijas, atsiranda problema, kaip jas apjungti, kad veiktų vieningai.
Tokiu atveju, galima naudotis saitynų tarnybų pagalba (angl. "Web services"), tačiau to neužtenka,
nes norima skirtingus funkcionalumus įgyvendinti skirtingų technologijų pagalba.
Tada dažnu atveju kuriamos mikroservisų (angl. \textit{„Microservices“}) sistemos. Remiantis šiuo modeliu būtų kuriami atskiri moduliai (taip pavadinsime atskirus sitemos vienetus), kurių
kiekvienas būtų atsakingas už tam tikrą funkcionalumą. Problema ta, kad moduliai yra skirtingi, tačiau jie turi sąveikauti tarpusavyje, o tai pasiekti gali būti sudėtinga.  
\subsectionnonum{Problema}
Kokį integracijų tipą rinktis mikroservisų architektūrose, norint įgyvendinti komunikaciją tarp skirtingų tarnybų (angl. \textit{„service“})?
\subsectionnonum{Darbo tikslas}
Palyginti skirtingus komunikacijų tipus, apžvelgti jų pranašumus ir trūkumus. Pateikti situacijų, kur vieni tipai yra pranašesni už kitus, pavyzdžių 
ir technologijų, šiems tipams realizuoti.
\subsectionnonum{Uždaviniai tikslui pasiekti}
\begin{enumerate}
	\item Remiantis literatūra apibūdinti, kas yra mikroservisai, kuo jie ypatingi, kaip jie projektuojami.
	\item Išskirti pagrindinius integracijų ir komunikavimo tipus ir jų savybes.
	\item Palyginti skirtingus komunikavimo būdus mikroservisų architektūrose.
	\item Pateikti konkrečius komunikavimo integracijų ir technologijų pavyzdžius.
	\item Pateikti rekomendacijas, kokiais atvejais, kokius tipus būtų geriau naudoti.
\end{enumerate}


\section{Kas yra mikroservisai}

\subsection{Monolitinės sistemos ir mikroservisų atsiradimas}
Pagal autorių Nicola Dragoni, Saverio Giallorenzo, Alberto Lluch Lafuente, Manuel Mazzara publikuotą straipsnį „Microservices: yesterday, today, and tomorrow“
\cite{Misc7} jau 1960-iais buvo susiduriama su problemomis susijusiomis su didžiulio masto PĮ kūrimu ir kaip tai projektuoti.
Buvo kuriama daug būdų kaip tai daryti, ir daug teorijų kaip tūrėtų atrodyti PĮ kodas, ir kaip projektuoti informacines sistemas.
Daug vėliau, apie 2000 metus, susiformavo sąvokos „Service-Oriented Computing“ (toliau SOC) ir „Service-Oriented Architecture” (toliau SOA), kurių idėja ir paradigmos buvo apie
tai, kad programa, arba kitaip pavadinus tarnyba, turėtų būti atsakingas už konkretaus resurso ar verslo logikos informaciją.
Šią tarnybą turi būti galima pasiekti su konkrečia technologija ir taip komunikuoti ir gauti informaciją. 
Taigi, remiantis jau ankščiau minėtu straipsniu „Microservices: yesterday, today, and tomorrow“ \cite{Mis7} iš SOC ir SOA kiek vėliau, susiformavo
mikroservisų idėja ir paradigmos. Pagal Martin Fowler ir James Lewis 2016-ais metais publikuotą straipsnį „Microservices“ \cite{Misc6}
šis terminas „mikroservisai“ buvo pirmą kartą diskutuotas 2011 metais, Venecijoje vykusiose PĮ kūrimo architektų
dirbtuvėse (angl. \textit{„workshop“}). Po metų, ta pati grupė architektų nusprendė, kad labiausiai tinkantis pavadinimas
šiam architektūriniam tipui yra mikroservisai (angl. \textit{„microservices“}). Po šiuo terminu slypi daug idėjų ir paradigmų,
tačiau pagrindinė mintis yra skaidymas didelės sistemos į mažas „granules“ ir mažas tarnybas.
Taip pat norint apibūdinti senas sistemas, kurios buvo nedalomos ir paleidžiamos vienu vykdomuoju paketu, atsirado terminas „monolitas“ (angl. \textit{„monolith“}).
Šis terminas ir buvo naudojamas Unix bendruomenės jau ilgą laiką iki mikroservisų susikūrimo.
Remiantis Eric Steven Raymon knyga „The Art of UNIX Programming“ \cite{Bk4}, kuri buvo išleista 2003 metais,
terminas „monolitas“ buvo naudojamas apibūdinti sistemoms, kurios yra per didelės. Taigi, šie du terminai
naudojami iki šiol apibūdinti informacinių sistemų architektūrinėms charakteristikoms.
\subsection{Mikroservisų pranašumai prieš monolitus}
Pagrindinius principus ir mikroservisų pranašumus detaliai išdėstė Sam Newman savo knygoje „Building Mircroservices: Designing Fine-Grained Systems“ \cite{Bk2}.
Šiame leidinyje autorius išgrynina keletą labai svarbių mikroservisų privalumų.
\subsubsection{Technologijų nevienalytiškumas (angl. \textit{„Technology Heterogeneity“})}
Kiekviena tarnyba turi atlikti skirtingas funkcijas ir turėti skirtingas atsakomybes. Norint pasiekti
geriausius rezultatus galima rinktis skirtingas technologijas, kurios būtų geriausiai pritaikytos konkrečiam uždaviniui spręsti.
Todėl mikroservisuose kiekvieną tarnybą galim projektuoti skirtingomis technologijomis, pavyzdžiui skirtingomis programavimo kalbomis.
\subsubsection{Atsparumas (angl. \textit{„Resilience“})}
Atsitikus problemai ir sugriuvus vienai sistemos komponentei, monolitinėje sistemoje tektų perkraudinėti arba taisyti visą sistemą.
Mikroservisų architektūroje stambių pasikeitimų būtų išvengta ir žlugtų tik vienos tarnybos veikimas. Tokiu atveju kitos tarnybos
apie tai nežinotų ir veiktų toliau, o norint ištaisyti problemą, užtektų sutaisyti ir perkrauti vieną tarnybą.
\subsubsection{Plečiamumas (angl. \textit{„Scaling“})}
Stambioje monolitinėje sistemoje visos plečiamumo ir efektyvumo problemos sprendžiamos kartu. Mikroservisų sistemoje
kiekvieną tokio tipo problemą sprendžiame atskirose tarnybose. Tokiu atveju tarnyboms, kurios reikalauja mažiau resursų
galima suteikti mažiau, o sunkesnioms ir mažiau efektyvioms tarnyboms išskirti daugiau.
Tačiau verta paminėti, kad mikroservisų sistemos plečiamumas ne visada yra geras, tai aprašyta
Omar Al-Debagy ir Peter Martinek straipsnyje „A Comparative Review of Microservices and Monolithic Architecture“ \cite{Misc3}
\subsubsection{Lengvas diegimas (angl. \textit{„Ease of Deployment“})}
Vystant PĮ dažnai susiduriama su diegimo problema. Su naujais funkcionalumais reikia iš naujo diegti naują PĮ versiją.
Monolitinėje sistemoje tenka iš naujo sudiegti visą sistemą, tačiau mikroservisų sistemoje galima sudiegti tik tas tarnybas, kurios yra susijusios
su pakeitimais.
\subsubsection{Organizacinis pasiskirstymas (angl. \textit{„Organizational Allignment“})}
Dažnai įmonėse prie informacinės sistemos dirba daug žmonių. Jie būna pasiskirstę komandomis ir turi skirtingas atsakomybes.
Mikroservisų sistemose galima išvengti komunikavimo incidentų ir kiekvienai komandai dirbti su skirtingomis tarnybomis.
Tokiu pavydžiu dirba stambi informacinių technologijų (toliau IT) įmonė „Netflix“.
\subsubsection{Kompozicija (angl. \textit{„Composability“})}
Įmonės dažnai susiduriama su problema, kai tas pats funkcionalumas reikalingas keliose informacinėse sistemose.
Mikroservisų architektūroje, kadangi sistema susideda iš mažų autonomiškų tarnybų, jas galima atskirti ir perpanaudoti
skirtingose sistemose, arba kitai sistemai suteikti prieigą tik prie konkrečių resursų, o ne visos sistemos.
\subsubsection{Optimizuotas pakeičiamumas (angl. \textit{„Optimizing for Replaceability“})}
Dirbant vidutinio dydžio arba didelėse imonėse dažnai susiduriama su problema, kai naudojama sena kodo bazė ir pasenusios technologijos.
Dažnai tokią sistemą reikia atnaujinti siekiant efektyvumo arba palaikymo paprastumo. Tokiu atveju, norint
atnaujinti bibliotekas arba technologijas, tenka iš naujo perprogramuoti dalį sistemos. Mikroservisų architektūros pagalba,
tai tampa žymiai papraščiau, kai užtenka perrašyti konkrečią tarnybą, neliečiant likusios informacinės sistemos.
\subsection{Monolitinių sistemų skaidymas į mikroservisus}
Paskutinį dešimtmetį tapo gan populiaru stambaus mąsto monolitus skaidyti į mikroservisus, tačiau tai nėra taip paprasta.
Visų pirma skaidant monolitą į atskiras tarnybas labai svarbu identifikuoti, kokios mažesnės tarnybos bus.
Tarnybų riboms apibrėžti panaudojau Micheal C. Feahters knygoje „Working Effectively with Legacy Code“ \cite{Bk5} apibrėžtą terminą
„siūlė“ (angl. \textit{„seam“}). Siūlė šiuo atveju yra kodo dalis, kuri yra izoliuota ir autonomiška.
Siūlės ir bus mūsų atskiri mikroservisai.
Autorė Susan J. Fowler savo knygoje „Production-Ready Microservices“ \cite{Bk1} aprašė patarimus ir žingsnius,
kaip reikėtų skaidyti monolitą į mikroservisus. Ji pamini, kad tai reikėtų daryti etapais:
\begin{enumerate}
	\item Monolitinę programą paleisti su tiek kopijų, kiek turėsime siūlių.
	\item Išskirstyti kvietimus į kopijas, pagal tai kokias siūles kopijos reprezentuoja.
	\item Išvalyti programų kopijas, paliekant tik siūlių funkcionalumą.
\end{enumerate}
Verta paminėti, kad po kiekvieno išvardinto žingsnio būtų atliekami regresiniai testavimai, kurie patikrintų
ar sistema veikia korektiškai.


\sectionnonum{Mikroservisų sistemos vidinių integracijų tipai}
Mikroservisų sistemos vidinių integracijų tipai


\section{Sinchroninės (angl. \textit{„synchronous“}) integracijos}

\subsection{Sinchroninių integracijų principas}
Sinchroninės integracijos dar gali būti apibūdinamos kaip „dviejų žmonių komunikavimas realiu laiku“ \cite{Bk6}. Su sinchroniniais komunikavimo pavyzdžiais dažnai susiduriame visi. Vienas iš tokių yra apsilankymas paprastoje internetinėje svetainėje.
Interneto naršyklėje suvedus puslapio pavadinimą, HTTP protokolo pagalba, mums iš serverio yra užkraunamas svetainės turinys ir atvaizduojamas.
Šis komunikavimo būdas yra paremtas užklausos/atsakymo (angl. \textit{„request/response“}) principu. Šiame modelyje egzistuoja
klientas (tai yra mūsų naršyklė) ir serveris (tai yra interneto svetainės savininkas arba įmonė teikianti svetainių talpinimo paslaugas).
Klientas siunčia užklausą serveriui prašydamas resurso, šiuo atveju tai yra mūsų norimos pamatyti svetainės turinio, ir laukia kol svetainė atiduos atsakymą į užklausą.
Serveris reaguodamas į kliento užklausą grąžina atsakymą, kuris turi būti arba svetainės turinys, arba grąžinama klaida, pavyzdžiui, kad toks resuras neegzistuoja arba 
klientas yra neautorizuotas šios svetainės lankytojas. Šis bendravimas yra perteiktas pateikta schema (\ref{img:synchronous-model} pav.):

\begin{figure}[H]
  \centering
  \includegraphics[scale=0.6]{img/synchronous-model}
  \caption{Sinchroninio komunikavimo schema.}
  \label{img:synchronous-model}
\end{figure}

Taigi šis modelis puikiai gali veikti ir mikroservisų sistemoje. Viena tarnyba siunčia užklausą į kitą tarnybą norėdama gauti
informaciją apie resursą arba inicijuoti, kokį nors veiksmą. Šis modelis ypatingas tuo, kad yra primityvus ir iškart užklausą išsiuntusi tarnyba
žinos ar sėkmingai pavyko atlikti norimą veiksmą. Šis modelis yra labai geras, kai tik įvykus sėkmingai užklausai leidžiame vartotojui vykdyti
kitas operacijas ir kitaip bendrauti su mūsų sistema, tai yra vartotojo autentifikacija ir autorizacija.

Siekant geriau išaiškinti, kaip veikia sinchroninis komunikavimas mikroservisų sistemose, pateikiama modelinė situacija, kur komunikavimas vyktų būtent
minėtu būdu. Įsivaizduokime, kad egzistuoja universiteto informacinė sistema, kuri yra suprojektuota pasiremiant mikroservisų
architektūriniu modeliu. Šioje sistemoje yra atskiros tarnybos atsakingos už studentų duomenų tvarkymą, dokumentų generavimą, elektroninių laiškų siuntimą, tvarkaraščių sudarymą 
ir saugojimą. Be šių tarnybų sistema turėtų turėti ir daug kitos paskirties mikroservisų, bet dabar aptarsime tik šiuos.
Taigi, sinchroninių komunikavimo modelių tarnyba atsakinga už studentų duomenų tvarkymą bus informuota apie naujo studento kūrimo inicijavimą.
Ši tarnyba savo duomenų bazėje sukuria naują įrašą ir vykdo tokius veiksmus eilės tvarka:

\begin{enumerate}
  \item Siunčiama užklausa į dokumentų generavimo tarnybą, kad būtų gauti sugeneruoti aktai apie naujo studento užregistravimą sistemoje, ir laukiama atsakymo. Dokumentų tarnyba sugeneravusi dokumentą grąžintų atsakymą apie sėkmingą
  dokumento sugeneravimą.
	\item Gavus atsakymą iš dokumentų tarnybos, siunčiama užklausa į tvarkaraščių generavimo tarnybą ir laukiama atsakymo. Tvarkaraščių generavimo tarnyba formuoja studento tvarkaraštį ir grąžina atsakymą apie sėkmingą resurso sukūrimą.
	\item Po sėkmingo tvarkaraščio sugeneravimo siunčiama užklausa į elektroninių laiškų tarnybą, siekiant informuoti naujo studento būsimus dėstytojus apie
  naują užsiėmimų dalyvį. Elektroninių laiškų tarnyba išsiuntusi visus laiškus grąžina atsakymą apie sėkmingai užbaigtą darbą.
  \item Be sistemos sutrikimų sėkmingai įvykdžius visus procesus, tarnyba atsakinga už studentų duomenų tvarkymą užbaigia naujo studento kūrimo procesą.
\end{enumerate}

Tokią veiksmų seką vaizdžiai parodo ši schema, kur veiksmai vyksta paeiliui, o ne lygiagrečiai (\ref{img:synchronous-microservices-scheme} pav.):

\begin{figure}[H]
  \centering
  \includegraphics[scale=0.6]{img/synchronous-microservices-scheme}
  \caption{Sinchroninio komunikavimo mikroservisuose schema.}
  \label{img:synchronous-microservices-scheme}
\end{figure}

\subsection{Sinchroninių integracijų technologijos}
Būdų realizuoti sinchroninius komunikavimo modelius yra daug. Šiuo metu populiariausi yra du:
\begin{itemize}
	\item RESTful saityno tarnybos.
	\item SOAP saityno tarnybos.
\end{itemize}
\break

Turint omenyje, kad REST yra tik architektūrinis stilius, o ne protokolas, ir jų nelabai galima lyginti \cite{Misc5}. Tačiau
remiantis Joni Makkonen magistrinio darbo „REST ir SOAP saityno tarnybų efektyvumo ir naudojamumo palyginimas“ \cite{MstrThs2} galima teigti, kad
REST yra pranašesnė ir šiame darbe išsiplėsime tik su šiuo architektūriniu stiliumi.
\break

REST saityno tarnybos užklausos struktūra yra paprasta. Ji susideda iš keletos atributų:
\begin{itemize}
	\item HTTP metodo, pvz.: \textit{„GET“, „POST“, „PUT“, „DELETE“}.
	\item Unikalaus resurso identifikatoriaus (arba adreso), pvz.: \textit{„http://mif.vu.lt/“}.
	\item Antraščių (angl. \textit{„Headers“}), pvz.: \textit{„Content-Type: application/json“}.
	\item Užklausos turinio (čia gali būti bet koks standartinis formatas, pvz.: \textit{„JSON“}).
	\item HTTP protokolo versija, pvz.: \textit{„HTTP/1.1“}.
\end{itemize}

REST atsakymo struktūra skiriasi nuo užklausos tik tuo, kad vietoje HTTP metodo, gaunamas HTTP statuso kodas. Tokio komunikavimo
per REST saityno tarnybas pavyzdį galime pamatyti šioje schemoje (\ref{img:Restful-scheme} pav.):

\begin{figure}[H]
  \centering
  \includegraphics[scale=0.6]{img/Restful-scheme}
  \caption{REST saityno tarnybų schema.}
  \label{img:Restful-scheme}
\end{figure}


\section{Asinchroninės (angl. “asynchronous”) integracijos}

\subsection{Asinchroninių integracijų principas}
Asinchroninių integracijų principas
\subsection{Asinchroninių integracijų technologijų tipai}
Asinchroninių integracijų technologijų tipai
\section{Skirtingų integracijų tipų privalumai ir trūkumai}
Skirtingų integracijų tipų privalumai ir trūkumai
\sectionnonum{Rezultatai ir išvados}

\subsectionnonum{Rezultatai}
Rezultatai
\subsectionnonum{išvados}
išvados

%% PAKEISTAS PAVADINIMAS Į 'Šaltiniai'
\printbibliography[heading=bibintoc, title=Šaltiniai]  % Šaltinių sąraše nurodoma panaudota
% literatūra, kitokie šaltiniai. Abėcėlės tvarka išdėstomi darbe panaudotų
% (cituotų, perfrazuotų ar bent paminėtų) mokslo leidinių, kitokių publikacijų
% bibliografiniai aprašai.  Šaltinių sąrašas spausdinamas iš naujo puslapio.
% Aprašai pateikiami netransliteruoti. Šaltinių sąraše negali būti tokių
% šaltinių, kurie nebuvo paminėti tekste.

% \sectionnonum{Sąvokų apibrėžimai}
Sąvokų apibrėžimai ir santrumpų sąrašas sudaromas tada, kai darbo tekste
vartojami specialūs paaiškinimo reikalaujantys terminai ir rečiau sutinkamos
santrumpos.


%\appendix  % Priedai
% Prieduose gali būti pateikiama pagalbinė, ypač darbo autoriaus savarankiškai
% parengta, medžiaga. Savarankiški priedai gali būti pateikiami ir
% kompaktiniame diske. Priedai taip pat numeruojami ir vadinami. Darbo tekstas
% su priedais susiejamas nuorodomis.

%\section{Neuroninio tinklo struktūra}
%\begin{figure}[H]
%    \centering
%    \includegraphics[scale=0.5]{img/MLP}
%    \caption{Paveikslėlio pavyzdys}
%    \label{img:mlp}
%\end{figure}

\end{document}
